\begin{abstract}
  Security protocols are crucial for protecting sensitive information and communications in today's digital age. Even a minor flaw in how these protocols are implemented can lead to serious consequences. Formal verification can prevent such flaws by providing security guarantees.

  Arquint et al. propose a generic and modular methodology to verify the security of protocol implementations.
  We extend their methodology to make it able to reason about ephemeral and sensitive data that must be quickly deleted.
  This enhancement allows us to verify strong security properties, such as forward secrecy and post-compromise security, for protocols that frequently renew their keys, such as the encrypted messaging protocol Signal.
  Our contributions encompass a conceptual expansion of their methodology and an extension of their Go library, which provides a way to verify Go protocol implementations relying on ephemeral data.
  A case study, featuring a Signal-like protocol implementation, showcases the practical application of our methodology.
\end{abstract}
