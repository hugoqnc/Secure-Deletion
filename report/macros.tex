\usepackage{listings, silver}
\usepackage{color, colortbl}
\usepackage{xspace}
\usepackage{hyphenat}
\usepackage[normalem]{ulem} %% Crossing out text (defines \sout)

\usepackage[inline]{enumitem} %% Provides inline lists, e.g. environment enumerate*
  \setlist*[enumerate]{label=(\arabic*)}

\usepackage{amsmath,amssymb,graphicx}
\usepackage{qsymbols}
\usepackage{stmaryrd} %% E.g. for \Mapsto
\usepackage{multirow}

\usepackage{suffix} %% Be able to declare starred macros, e.g. \foo*
\usepackage{xparse}

\newcommand*{\ListingsFontSize}{\fontsize{8.0}{8.3}\selectfont}
\newcommand*{\SilFont}{\ttfamily\ListingsFontSize}





% colors (see https://latexcolor.com/)
\definecolor{light-gray}{gray}{0.9}
\definecolor{darkgreen}{rgb}{0.0,0.7,0.0}
\definecolor{darkred}{rgb}{0.55, 0.0, 0.0}
\definecolor{persianblue}{rgb}{0.11, 0.22, 0.73}
\definecolor{navyblue}{rgb}{0.0, 0.0, 0.5}
\definecolor{darkpowderblue}{rgb}{0.0, 0.2, 0.6}
\definecolor{frenchblue}{rgb}{0.0, 0.45, 0.73}
\definecolor{burntorange}{rgb}{0.8, 0.33, 0.0}

\definecolor{rowShade}{gray}{0.85}



%% ****************************************************************************

\newcommand*{\Figref}[1]{Fig.~\ref{fig:#1}}
\newcommand*{\figref}{\Figref}
\newcommand*{\Secref}[1]{Sec.~\ref{sec:#1}}
\newcommand*{\secref}{\Secref}
\newcommand*{\Appref}[1]{App.~\ref{sec:#1}}
\newcommand*{\appref}{\Appref}
\newcommand*{\Lineref}[1]{Line~\ref{line:#1}}
\newcommand*{\lineref}[1]{line~\ref{line:#1}}
\newcommand*{\Linerange}[2]{Lines~\ref{line:#1}-\ref{line:#2}}
\newcommand*{\linerange}[2]{lines~\ref{line:#1}-\ref{line:#2}}
% \newcommand*{\Sref}[1]{\hyperref[#1]{\S\ref*{#1}}}

%% ****************************************************************************

\newcommand*{\Ie}{I.e.\xspace}
\newcommand*{\ie}{i.e.\xspace}
\newcommand*{\Eg}{E.g.\xspace}
\newcommand*{\eg}{e.g.\xspace}
\newcommand*{\Cf}{Cf.\xspace}
\newcommand*{\cf}{cf.\xspace}
\newcommand*{\Wrt}{W.r.t.\xspace}
\newcommand*{\wrt}{w.r.t.\xspace}

\newcommand*{\naive}{na\"{\i}ve\xspace}
\newcommand*{\Naive}{Na\"{\i}ve\xspace}

%% ****************************************************************************

% \newcommand*{\ListingsFontSize}{\fontsize{8.0}{8.3}\selectfont}
% \newcommand*{\SilverFont}{\ttfamily\ListingsFontSize}
% \usepackage{silver}

%% listings
\lstset{numberstyle=\small} % size of line numbers
\lstset{mathescape=true}

\newcommand{\vcode}[1]{\vipercode{#1}}



% \lstset{
%   % basicstyle={\small\fontfamily{cmtt}\selectfont}
%   % basicstyle={\ttfamily\selectfont\footnotesize},
%   % commentstyle={\ttfamily\selectfont\footnotesize},
%   tabsize=2,
%   mathescape=true,
%   language=silver
%   % basicstyle=\ttfamily
% }

\newcommand*{\GobraFont}{\SilFont}
\newcommand*{\GobraFontSize}{\ListingsFontSize}
\newcommand*{\InlineGobraFontSize}{\small}

%% TODO: Add keywords (can we take them from Felix' thesis?)
\lstdefinelanguage{gobra}{
  language=go,
  sensitive=true,
  morecomment=[l]{//},
  morecomment=[s]{/*}{*/},
  morekeywords=[1]{ %% Keywords of the programming language subset
    pred, implements
  },
  morekeywords=[2]{ %% Keywords of the specification language subset
    requires, ensures, invariant, req, ens, pure
  },
  morekeywords=[3]{ %% Keywords of the proof language subset
    fold, unfold, unfolding, in, ghost,
    assume, assert, inhale, exhale
  },
  basicstyle={\GobraFont\GobraFontSize},
  commentstyle={\color[HTML]{747678}\textit},
  keywordstyle={[1]\color[HTML]{0005FF}},%\bfseries
  keywordstyle={[2]\color{burntorange}},%\bfseries
  keywordstyle={[3]\color[HTML]{EC008C}},%\bfseries
  mathescape=true,
  escapechar=§,
  moredelim=**[is][\normalfont\itshape]{'}{'} 
    %% Should use \var rather than \normalfont\itshape, but that doesn't work. 
    % Probably because \var expects an argument.
}

\lstdefinelanguage{myViper}{
  language=silver,
  sensitive=true,
  basicstyle={\GobraFont\GobraFontSize},
  mathescape=true,
  escapechar=§,
  moredelim=**[is][\normalfont\itshape]{'}{'} 
    %% Should use \var rather than \normalfont\itshape, but that doesn't work. 
    % Probably because \var expects an argument.
}

\lstdefinelanguage{plaintext}{
  language=,
  % basicstyle={\GobraFont\GobraFontSize},
  % commentstyle={\GobraFont\GobraFontSize\color{red}},
  keywordstyle=\bfseries,
  ndkeywordstyle=\bfseries,
  morecomment=[l]{//},
  mathescape=true
}

\makeatletter
\lst@Key{countblanklines}{true}[t]%
    {\lstKV@SetIf{#1}\lst@ifcountblanklines}

\lst@AddToHook{OnEmptyLine}{%
    \lst@ifnumberblanklines\else%
       \lst@ifcountblanklines\else%
         \advance\c@lstnumber-\@ne\relax%
       \fi%
    \fi}
\makeatother

\lstnewenvironment{gobra}[1][]{
  \lstset{
    language=gobra,
    floatplacement={tbp},
    captionpos=b,
    frame=lines,
    numbers=left,
    numberblanklines=false,
    xleftmargin=8pt,
    xrightmargin=8pt,
    countblanklines=false,
    numberstyle=\scriptsize,
    #1
  }
}{}


\lstnewenvironment{myViper}[1][]{
  \lstset{
    language=myViper,
    floatplacement={tbp},
    captionpos=b,
    frame=lines,
    numbers=left,
    numberblanklines=false,
    xleftmargin=8pt,
    xrightmargin=8pt,
    #1
  }
}{}

%% Use similar to \lstinline itself, e.g \gobra!x := a.f!, i.e. with a
%% surrounding delimiter.
%% http://tex.stackexchange.com/questions/100427/
\newcommand*{\inlgobra}{%
  \lstinline[%
    language=gobra,%
    columns=fixed,%
    basicstyle={\GobraFont\InlineGobraFontSize},%
    mathescape=true,%
    commentstyle={\color{black}\InlineGobraFontSize},
    keywordstyle={[1]\color{black}\InlineGobraFontSize},
    keywordstyle={[2]\color{black}\InlineGobraFontSize},
    keywordstyle={[3]\color{black}\InlineGobraFontSize},
    keywordstyle={[4]\color{black}\InlineGobraFontSize}
  ]%
}
\newcommand*{\gl}{\inlgobra}

\newcommand*{\inlmyviper}{%
  \lstinline[%
    language=myViper,%
    columns=fixed,%
    basicstyle={\GobraFont\InlineGobraFontSize},%
    commentstyle={\color{black}\InlineGobraFontSize},
    keywordstyle={[1]\color{black}\InlineGobraFontSize},
    keywordstyle={[2]\color{black}\InlineGobraFontSize},
    keywordstyle={[3]\color{black}\InlineGobraFontSize},
    keywordstyle={[4]\color{black}\InlineGobraFontSize}
  ]%
}
\newcommand*{\vl}{\inlmyviper}

%% Ensure that \texttt and \inlgobra use the same font in the same size.
%% Important, because it is sometimes seemingly impossible to use \inlgobra 
%% (in general, verbatim text) in certain situations, in which case one can
%% fallback to \texttt.
\usepackage{letltxmacro} %% Provides \LetLtxMacro
\LetLtxMacro\oldttfamily\ttfamily
\DeclareRobustCommand{\ttfamily}{\oldttfamily\InlineGobraFontSize}

%% ****************************************************************************

% Encoding

\newcommand{\encodeT}[1]{\llbracket #1 \rrbracket}
\newcommand{\shared}[0]{@}
\newcommand{\exclusive}[0]{\bullet}
\newcommand{\encAs}[0]{\triangleq}

\newcommand{\implementationProof}{_{\textsc{Proof}}}